\chapter{Conclusions}\label{chap:conclusions}

Throughout this work, several methods have been theoretically justified and successfully applied to quantify the prediction uncertainty both for regression and time series problem. These strategies are distribution-free \& model-agnostic and stem from the notion of "\textit{conformalizing}" predictions to data and using the residuals to understand the errors distribution. That is why they are grouped within the so-called "conformal prediction" (CP) methodologies.\\

Even though CP paradigm was classically applied only under "\textit{data exchangeability}" conditions, this work has reviewed some of the most recent \& non-trivial efforts to enable CP when this hypothesis is not fulfilled. 

In particular, while SCP, CV$+$, J$+$aB \& CQR were studied for the exchangeable case; regarding the time series case, "\textit{EnbPI}" (\cite{chenxu2021a}) was presented as the strategy to effectively obtain prediction intervals with statistically valid coverage.

While all the former strategies were successfully applied to practical case, generally providing valid intervals, below the more fine-grained conclusions are listed:
\begin{itemize}
    \item \textbf{Exchangeable case} (regression problem):
    \begin{itemize}
        \item The best strategies, \textbf{decreasingly ordered} by:
        \begin{itemize}
            \item \textit{Statistical efficiency} are: CQR, SCP, CV$+$, J$+$aB. This is fulfilled independently of $\a$.
            \item \textit{Computational efficiency} are: SCP, CQR, CV$+$, J$+$aB.
            \item \textit{Predictive power} are: CV$+$ \& J$+$aB, SCP, CQR.
            \item "\textit{Informativeness}" (coverage-width ratio) are: J$+$aB, SCP, CQR, CV$+$.
            \item \textit{Adaptability} are: CQR, CV$+$, J$+$aB (slight to none). Contrarily, SCP intervals are not adaptive at all. 
        \end{itemize}
        \item \textbf{CQR} seems the best \textbf{strategy} to achieve the best \textbf{marginal} \& \textbf{conditional coverage}, when dataset is large enough. 

        Thus, it may result suitable when a conservative and statistical efficient tool is needed.
        \item \textbf{J$+$aB} seems the best \textbf{strategy} to achieve the best \textbf{informative intervals} (maximizing predictive power, while minimizing intervals width), at expenses of no-adaptability \& losing some coverage. 
        
        Thus, it may result suitable when more reckless guesses can be afforded and low training \& inference times are not a requirement.
    \end{itemize}
    \item \textbf{Non-exchangeable case} (time series problem):
    \begin{itemize}
        \item \textbf{EnbPI} is a \textbf{suitable option} to provide valid intervals for \textbf{time series problems}.
        \item In general, and particularly when there might be strong shifts in data, EnbPI's intervals adjustment using test residuals (its "partial fit" feature) is of crucial importance.
        \item This "\textbf{partial fit}" option will not only allow the intervals' coverage \textbf{recover from change points}, but also will allow all the issued \textbf{intervals} to be \textbf{adaptive}.
    \end{itemize}
\end{itemize} 

\section{Further research}\label{sec:further-research}

There a huge number of relevant other inquiries and research lines which could extend this work, but were out of the scope of this thesis. 
Below, some of them are reviewed: 
\begin{itemize}
\item Leverage cross-validation folds in the CQR strategy, instead of a simple train-test split of the dataset, to improve the predictive power and reduce the need of a large dataset. 
\item Implement other contemporary CP methodologies for time series problems, such as \textit{Adaptive Conformal Inference}, ACI (\cite{gibbs2021}) and the more recent \textit{Hopfield Conformal Prediction Trees}, HopCPT (\cite{auer2023}); in order to compare performance differences and their suitability.
\item Extend all these methods to the multi-dimensional output variables' case, to broaden their applicability to multi-output regression \& time series problems.
\item Apply the former methodologies to classification problems (discrete target variables) and discuss whether similar conclusions to their continuous counter-part can be drawn.
\end{itemize}